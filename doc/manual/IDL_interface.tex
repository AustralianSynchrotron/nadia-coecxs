
\subsection{NADIA\_ADD\_COMPLEX\_CONSTRAINT\_REGION}
\begin{verbatim}

 NAME:
       NADIA_ADD_COMPLEX_CONSTRAINT_REGION

 PURPOSE: For use with FresnelCDI reconstruction. Calling this
       procedure will constrain the transmission function magnitude
       and phase using the method from the paper "Use of a complex
       constraint in coherent diffractive imaging", J. N. Clark
       et. al., 2010. The notation used there is also used
       here. Users should have knowledge of this or similar papers
       before using the procedure.

       A section of the reconstructed transmission function will be
       updated according to a restriction on the value of c (c =
       beta/delta). If the material is of a known element, then c can
       be fixed to the know value, and the phase and magnitude of the
       transmission function updated accordingly. Alternatively, c
       can be left to float, and calculated for each iteration from a
       mean over the defined region. The parameters alpha1 and alpha2
       are used to control the strength of the constraint. This
       procedure should be called once before each region of
       homogenious material.

 CALLING SEQUENCE:

       NADIA_ADD_COMPLEX_CONSTRAINT_REGION, region, alpha1, alpha2 [, fixed_c]

 INPUTS:

       region:
             A 2 array of doubles which is used to indicate which
             pixels the constraint should be applied to. 0 -
             indicated the array element does not belong to the
             region. Any other value indicated that it does.

       alpha1:
             Constraint strength parameter for the amplitude (double
             type).

       alpha2:
             Constraint strength parameter for the phase (double
             type).
             
       fixed_c:
             Optional parameter to fix the value of c = beta/delta
             (double type).

 EXAMPLE:
       Setting a complex constraint for two image regions.
       For the second, the value of c=beta/delta is fixed.

       r1 =  nadia_read_tiff(1024,1024,'region_1.tiff')
       r2 =  nadia_read_tiff(1024,1024,'region_2.tiff')

       delta = 6.45e-4;
       beta = 1.43e-4;

       nadia_add_complex_constraint_region, r1, 0.5, 0.5
       nadia_add_complex_constraint_region, r2,   1,   0, beta/delta

\end{verbatim}



  
 
\subsection{NADIA\_APPLY\_SHRINKWRAP}
\begin{verbatim}

 NAME:
       NADIA_APPLY_SHRINKWRAP

 PURPOSE:
       Apply the shrinkwrap algorithm. The current exit-surface-wave
       magnitude is used to update the support; it is convoluted with
       a Gaussian and then a threshold is applied. You can use the
       nadia_get_support function to see how the support have been
       modified after calling this procedure.

 CALLING SEQUENCE:

	NADIA_APPLY_SHRINKWRAP [,gauss_width, threshold ]

 INPUTS:

       gauss_width: 
             The width (1-standard deviation. in pixels) of the Gaussian
             used for smearing. If this parameter is not passed, 
             a width of 1.5 pixels is used.

       threshold:
             All pixels which are below the threshold are set to zero 
             (outside the support). The threshold should be given as
             a fraction of the maximum pixel value (in double format). 
             If this parameter is not passed a threshold of 0.1 (10%)
             is used.

 EXAMPLE:
       Perform 1000 iterations in total, applying shrink-wrap at the 400th iteration:

       ....
       a = NADIA_ITERATE(400)
       NADIA_APPLY_SHRINKWRAP
       a = NADIA_ITERATE(600)

\end{verbatim}


\subsection{NADIA\_APPLY\_SUPPORT}
\begin{verbatim}

 NAME:
       NADIA_APPLY_SUPPORT

 PURPOSE:
       Apply the support constraint to the given complex array. All
       elements outside the support with be reset to zero. Elements
       within the support will be left as they are. The support must
       have been previously set using either one of the NADIA_INIT 
       functions or NADIA_SET_SUPPORT.

       This function is called when using (NADIA_ITERATE) so generally
       won't need to be call explicitly. An exception to this is if
       the user wishes to extend their reconstruction with an
       addition constraint.


 CALLING SEQUENCE:

	result = NADIA_APPLY_SUPPORT( complex_array [,/SUPPRESS_DISPLAY] )

 INPUTS:

       complex_array:
             A 2D array of COMPLEX values which represents a wave in
             either the sample plane (for planar and Fresnel
             reconstruction) or the zone-plate plane for Fresnel 
             white-field reconstruction.

 KEYWORD PARAMETERS:

       /SUPPRESS_DISPLAY:
             Do not display the result on the screen. This maybe useful
             if this function is used within a for loop. 

 OUTPUTS:

       result:
             A COMPLEX 2D array after the support is applied.

 EXAMPLE:
       See the example for NADIA_PROPAGATE_FROM_DETECTOR

\end{verbatim}




\subsection{NADIA\_CLEAR\_MEMORY}
\begin{verbatim}

 NAME:
       NADIA_CLEAR_MEMORY

 PURPOSE:
       Clean-up after a reconstruction has been performed. This 
       procedure should be called at the very end of a program.
       It will free up the memory that was allocated when one of 
       the "NADIA_INIT_.." methods was called.

 CALLING SEQUENCE:

       NADIA_CLEAR_MEMORY

 EXAMPLE:
       NADIA_CLEAR_MEMORY

\end{verbatim}

\subsection{NADIA\_GET\_BEST\_RESULT}
\begin{verbatim}

 NAME:
       NADIA_GET_BEST_RESULT

 PURPOSE:
       Get the best (lowest error) result found during the reconstruction.
       The result is returned and the magnitude of the result is
       displayed as a 512x512 pixel image on the screen.

 CALLING SEQUENCE:

       result = NADIA_GET_BEST_RESULT()

 OUTPUTS:

       result:
             A 2D array of COMPLEX variables. For planar and Fresnel
             reconstruction this will be the  exit-surface-wave for
             the sample. For Fresnel white-field reconstruction, this
             will be the complex white-field in the detector plane.


 EXAMPLE:
       Perform 1000 iterations and get the lowest error result.
       ....
       a = NADIA_ITERATE(1000)
       a = NADIA_GET_BEST_RESULT()

\end{verbatim}

\subsection{NADIA\_GET\_ERROR}
\begin{verbatim}

 NAME:
       NADIA_GET_ERROR

 PURPOSE:
       Get the error metric. This is defined as the difference
       between the estimated diffraction and the actual diffraction
       pattern and is calculated as:
           sum over pixels of ( M - sqrt(I) ) ^2 / sum(I) 
       Where I is the detector data and M is the magnitude of the
       estimate in the detector plane. Note that due to the way this
       quantity is calculated, it actually corresponds to the
       previous estimate rather than the current iteration.


 CALLING SEQUENCE:

       error = NADIA_GET_ERROR()

 OUTPUTS:

       error:
             The error. 

 EXAMPLE:

       a = NADIA_GET_ERROR()

\end{verbatim}

 
\subsection{NADIA\_GET\_INTENSITY\_AUTOCORRELATION}
\begin{verbatim}

 NAME:
       NADIA_GET_INTENSITY_AUTOCORRELATION

 PURPOSE:
       Get the autocorrelation function from the intensity data.
       This method is only useful for planar CDI reconstruction.

 CALLING SEQUENCE:

       a = NADIA_GET_INTENSITY_AUTOCORRELATION()

 OUTPUTS:
       a:
             The autocorrelation function (a 2D array of doubles).

 EXAMPLE:
       a = NADIA_GET_INTENSITY_AUTOCORRELATION()

\end{verbatim}

\subsection{NADIA\_GET\_ROUND\_SUPPORT}
\begin{verbatim}

 NAME:
       NADIA_GET_ROUND_SUPPORT

 PURPOSE: 
       This function will return a simple array with a central
       circular region where each element has the value 1.0. Outside
       the value returned is zero. This function has been written to
       allow each create of the support array. For example when
       reconstructing the white-field for Fresnel CDI reconstruction.
       Note that it does not actually call any method from the NADIA
       software library.

 CALLING SEQUENCE:

	NADIA_GET_ROUND_SUPPORT, nx, ny, radius

 INPUTS:

       n_x: 
             The number of pixels in the horizontal direction of the
             output array.
       n_y: 
             The number of pixels in the vertical direction of the
             output array.
       radius: 
             The radius of the circle in pixels.

 RETURN:
       A two dimensional array which can be used to set the support
       for any of the CDI reconstructions.

 EXAMPLE:

       nadia_get_round_support, 1024, 1024, 0.25*1024

\end{verbatim}

 
\subsection{NADIA\_GET\_SUPPORT}
\begin{verbatim}

 NAME:
       NADIA_GET_SUPPORT

 PURPOSE:
       Get the support. This maybe useful to see how shrinkwrap has
       effected the support. The result is displayed as a 512x512 
       pixel image on the screen.

 CALLING SEQUENCE:

       support = NADIA_GET_SUPPORT()

 OUTPUTS:

       support:
             The support. A 2D array of doubles. A pixel with value below 1
             is outside the support and 1 and above is inside the support.

 EXAMPLE:
       View the support after applying shrink-wrap.
 
       NADIA_APPLY_SHRINKWRAP
       a = NADIA_GET_SUPPORT()

\end{verbatim}

\subsection{NADIA\_GET\_TRANSMISSION\_FUNCTION}
\begin{verbatim}

 NAME:
       NADIA_GET_TRANSMISSION_FUNCTION

 PURPOSE:
       Get the transmission function from the current estimate of the
       exit-surface-wave. The magnitude of the result will be
       displayed on the screen. Note that this functions
       is only available for Fresnel CDI reconstruction.

 CALLING SEQUENCE:

       result = NADIA_GET_TRANSMISSION_FUNCTION()

 OUTPUTS:

       result:
             The transmission function for the sample. A 2D array of
             COMPLEX variables is returned. 

 EXAMPLE:
       a = NADIA_GET_TRANSMISSION_FUNCTION()

\end{verbatim}

 
\subsection{NADIA\_INITIALISE\_ESW}
\begin{verbatim}

 NAME:
       NADIA_INITIALISE_ESW

 PURPOSE:
       Initialise the exit-surface-wave guess. The initialisation
       will depend on the reconstruction type (see the procedures
       which begin "NADIA_INIT_" for a description). The "INIT" procedure  
       will call this procedure if no starting guess is provided. 
       It is useful if you wish to run the same reconstruction
       several time with a different random starting point, or if 
       you wish to reset the reconstruction back to the original guess.


 CALLING SEQUENCE:

	NADIA_INITIALISE_ESW, seed

 INPUTS:

       seed: 
             Seed for the random number generator used to initialise
             the guess. It should be an integer. This is ignored in 
             the case of Fresnel CDI.


 EXAMPLE:

       nadia_initialise_esw, 6

\end{verbatim}


\subsection{NADIA\_INIT\_FRESNEL}
\begin{verbatim}

 NAME:
       NADIA_INIT_FRESNEL

 PURPOSE:
       Set-up a Fresnel CDI reconstruction. This will
       initialise the reconstruction using a previously reconstructed
       white-field, detector data, sample support and experimental 
       parameters. Some defaults will be set and memory will be
       allocated ready for reconstructing the sample
       exit-surface-wave. It is necessary to call this procedure before
       attempting to call any of the reconstruction methods
       (e.g. before setting the algorithm or calling NADIA_ITERATE).

       Calling this procedure will initialise the reconstruction algorithm
       to the error-reduction with a relaxation parameter of 0.9.

 CALLING SEQUENCE:

	NADIA_INIT_FRESNEL, data, support, white-field, beam_wavelength,
	                  focal_detector_length, focal_sample_length, 
                         pixel_size [, normalisation, starting_point ]

 INPUTS:

       You may use any length units for the experimental parameters
       below, as long as all quantities are given in the same units.

	data: 
             The detector data with the sample in place. It should be
             in the form of a 2D array.

       support: 
             A 2D array of integers or doubles which give the sample support.
             Values or 1 or greater are considered inside
             the support. All others are considered to be
             outside the support.

       white-field:
             A COMPLEX 2D array of the reconstructed white-field in
             the detector plane. This can be recovered using 
             NADIA_INIT_FRESNEL_WF followed by NADIA_ITERATE.

       beam_wavelength:
             The beam wavelength.

       focal_detector_length:
             The distance between the focal point and the detector.

       focal_sample_length:
             The distance between the focal point and the sample.

       pixel_size:
             The side length of one detector pixel.

       normalisation: 
             The factor to scale the white-field before
             performing FCDI. If this parameter is excluded, the
             ratio of the square-root of the intensity data and the
             white-field magnitude is used as the normalisation.

       starting_point: 
             As an option you may supply an initial 
             guess of the exit-surface-wave for the sample. 
             This maybe useful, for example, if you wish to 
             start from the end point of a previous run. The
             format of this parameter much be a 2D array of
             COMPLEX variables. If this parameter is not supplied,
             the initialisation described in Harry's review paper: 
             page 29. (in particular e.q. 137) is used.

 EXAMPLE:

        nadia_init_fresnel, my_data, my_supports, my_white-field, $
                          4.892e-10, 0.9078777, 2.16e-3, $
                          13.5e-6

\end{verbatim}

\subsection{NADIA\_INIT\_FRESNEL\_WF}
\begin{verbatim}

 NAME:
       NADIA_INIT_FRESNEL_WF

 PURPOSE:
       Set-up a Fresnel white-field CDI reconstruction. This will
       initialise the reconstruction with the white-field intensity, 
       zone-plate support and experimental parameters. Some defaults 
       will be set and memory will be allocated ready for 
       reconstructing the white-field (phase and magnitude) in the
       detector plane. It is necessary to call this procedure before
       attempting to call any of the reconstruction methods
       (e.g. before setting the algorithm or calling NADIA_ITERATE).

 CALLING SEQUENCE:

	NADIA_INIT_FRESNEL_WF, data, support, beam_wavelength,
                            zone_focal_length, focal_detector_length,
                            pixel_size [,starting_point]

 INPUTS:

       You may use any length units for the experimental parameters
       below, as long as all quantities are given in the same units.

	data: 
             The white-field illumination. It should be
             in the form of a 2D array.

       support: 
             A 2D array of integers or doubles giving the zone-plate support.
             Values or 1 or greater are considered inside
             the support. All others are considered to be
             outside the support.

       beam_wavelength:
             The beam wavelength.

       zone_focal_length:
             The distance between the zone plate and the focal point.

       focal_detector_length:
             The distance between the focal point and the detector.

       pixel_size:
             The side length of one detector pixel.

       starting_point: 
             As an option you may supply an initial guess of the 
             white-field. This may be useful, for example, if you 
             wish to start from the end point of a previous run. The
             format of this parameter must be a 2D array of
             COMPLEX variables. If this parameter is not supplied,
             the starting point is initialised to be zero outside
             the support, a random number inside the support 
             for the magnitude and zero for the phase.

 EXAMPLE:

        An example of loading two 2D arrays from file and using
        them to initialise the white-field reconstruction for FCDI:

        my_support = nadia_read_tiff(1024,1024,'support.tiff')
        my_data = nadia_read_tiff(1024,1024,'data.tiff')
        nadia_init_fresnel_wf, my_data, my_support, $
                   4.892e-10, 16.353e-3, 0.9078777,13.5e-6

\end{verbatim}

\subsection{NADIA\_INIT\_PLANAR}
\begin{verbatim}

 NAME:
       NADIA_INIT_PLANAR

 PURPOSE:
       Set-up a planar CDI reconstruction. This will
       initialise the reconstruction with the data
       and support. Some defaults will be set and
       memory will be allocated ready for reconstruction.
       It is necessary to call this procedure before
       attempting to call any of the reconstruction methods
       (e.g. before setting the algorithm or 
       calling NADIA_ITERATE).

       Calling this procedure will initialise the reconstruction 
       algorithm to hybrid-input-output with a relaxation parameter of 0.9.


 CALLING SEQUENCE:

	NADIA_INIT_PLANAR, Data, Support [,Starting_point]


 INPUTS:

	Data: 
             The detector illumination. It should be
             in the form of a 2D array

       Support: 
             A 2D Array giving the sample support.
             Values or 1 or greater are considered inside
             the support. All others are considered to be
             outside the support.

       Starting_point: 
             As an option you may supply an initial 
             guess of the exit-surface-wave for the sample. 
             This maybe useful, for example, if you wish to 
             start from the end point of a previous run. The
             format of this parameter much be a 2D array of
             COMPLEX variables. If this parameter is not supplied,
             the starting point is initialised to be zero outside
             the support and a random number inside the support, 
             for both the magnitude and phase.

 EXAMPLE:
        An example of loading two 2D arrays from file and using
        them to initialise the planar reconstruction:

        my_support = nadia_read_tiff(1024,1024,'planar_support.tiff')
        my_data = nadia_read_tiff(1024,1024,'planar_data.tiff')
        NADIA_INIT_PLANAR, my_data, my_support


\end{verbatim}






  
 
\subsection{NADIA\_ITERATE}
\begin{verbatim}

 NAME:
       NADIA_ITERATE

 PURPOSE:
       Perform the iterative reconstruction. The number of iterations
       to perform should be given and the result of the final
       iteration is returned. For planar and Fresnel CDI this will be
       the exit surface wave of the sample. For Fresnel white-field
       reconstruction it will be the white-field at the detector surface.
       The magnitude of the result is also displayed on the screen as
       a 512x512 pixel image. The iteration number and corresponding
       error (see NADIA_GET_ERROR) will be printed on the screen.

       NADIA_ITERATE may be called several time and the reconstruction
       will start from where is ended. i.e. calling NADIA_ITERATE(50),
       followed by a second NADIA_ITERATE(50) is equivalent to calling
       NADIA_ITERATE(100).

       During the reconstruction, the best (lowest error) result is
       also stored. It maybe retrieved by calling NADIA_GET_BEST_RESULT.
 

 CALLING SEQUENCE:

	result = NADIA_ITERATE([iterations])

 INPUTS:

       iterations: 
             The number of iterations to perform (an integer). 
             If left empty, one iteration is performed.

 OUTPUTS:

       result:
             A COMPLEX 2D array. For planar and Fresnel CDI this will be
             the exit surface wave of the sample. For Fresnel white-field
             reconstruction it will be the white-field at the detector surface.


 EXAMPLE:

       my_result = nadia_iterate(100)

\end{verbatim}






  
 
\subsection{NADIA\_PRINT\_ALGORITHM}
\begin{verbatim}

 NAME:
       NADIA_PRINT_ALGORITHM

 PURPOSE:
       Output the form of the current algorithm to the screen. 
       It will be written in terms of the support and intensity
       projection operators.


 CALLING SEQUENCE:
       NADIA_PRINT_ALGORITHM


 EXAMPLE:
       NADIA_SET_ALGORITHM, 'RAAR'
       NADIA_PRINT_ALGORITHM

\end{verbatim}



  
 
\subsection{NADIA\_PROPAGATE\_FROM\_DETECTOR}
\begin{verbatim}

 NAME:
       NADIA_PROPAGATE_FROM_DETECTOR

 PURPOSE:
       Propagate the given wave field from the detector plane to the 
       sample plane (planar and Fresnel) or zone-plate plane (Fresnel
       white-field reconstruction). The result will be returned and
       displayed on the screen by default.

       This function is called when using (NADIA_ITERATE) so generally
       won't need to be call explicitly. An exception to this is if
       the user wishes to extend the reconstruction with an addition
       constraint (see the example below).


 CALLING SEQUENCE:

	result = NADIA_PROPAGATE_FROM_DETECTOR( complex_array [,/SUPPRESS_DISPLAY] )

 INPUTS:

       complex_array:
             A 2D array of COMPLEX values which represents a wave in
             the detector plane.

 KEYWORD PARAMETERS:

       /SUPPRESS_DISPLAY:
             Do not display the result on the screen. This maybe useful
             if this function is used within a for loop. 

 OUTPUTS:

       result:
             A COMPLEX 2D array which is either the field in the
             sample plane (for planar and Fresnel reconstruction) or
             the zone-plate plane for Fresnel white-field reconstruction.

 EXAMPLE:
       Performing the reconstruction with a new constraint applied in the
       sample plane (e.g. called "NEW_SUPPORT"):

              FOR K = 0, 100 DO BEGIN 
                  a = NADIA_PROPAGATE_TO_DETECTOR(a,/SUPPRESS_DISPLAY)
                  a = NADIA_SCALE_INTENSITY(a,/SUPPRESS_DISPLAY)
                  a = NADIA_PROPAGATE_FROM_DETECTOR(a,/SUPPRESS_DISPLAY)
                  a = NADIA_APPLY_SUPPORT(a,/SUPPRESS_DISPLAY)
                  a = NEW_SUPPORT(a)
              ENDFOR 


\end{verbatim}






  
 
\subsection{NADIA\_PROPAGATE\_TO\_DETECTOR}
\begin{verbatim}

 NAME:
       NADIA_PROPAGATE_TO_DETECTOR

 PURPOSE:
       Propagate the given wave field to the detector plane from the 
       sample plane (planar and Fresnel) or zone-plate plane (Fresnel
       white-field reconstruction). The result will be returned and
       displayed on the screen by default.

       This function is called when using (NADIA_ITERATE) so generally
       won't need to be call explicitly. An exception to this is if
       the user wishes to perform simulation or to extend their
       reconstruction with an addition constraint (see the previous
       example).


 CALLING SEQUENCE:

	result = NADIA_PROPAGATE_TO_DETECTOR( complex_array [,/SUPPRESS_DISPLAY] )

 INPUTS:

       complex_array:
             A 2D array of COMPLEX values which represents a wave in
             either the sample plane (for planar and Fresnel
             reconstruction) or the zone-plate plane for Fresnel 
             white-field reconstruction.

 KEYWORD PARAMETERS:

       /SUPPRESS_DISPLAY:
             Do not display the result on the screen. This maybe useful
             if this function is used within a for loop. 

 OUTPUTS:

       result:
             A COMPLEX 2D array which is the field in the detector plane.

 EXAMPLE:
       Performing Fresnel simulation (assuming you already have a
       complex white-field called "white_field" and the transmission
       function of the object, "trans").

       white_field_at_sample = NADIA_PROPAGATE_FROM_DETECTOR(white_field)
       esw_at_sample = trans*white_field_at_sample
       esw_at_detector = NADIA_PROPAGATE_TO_DETECTOR(esw_at_sample)
       diffraction_pattern = abs(esw_at_detector)^2

\end{verbatim}






  
 
\subsection{NADIA\_READ\_CPLX}
\begin{verbatim}

 NAME:
       NADIA_READ_CPLX

 PURPOSE:
       Read a binary file of fftw complex numbers. This is
       useful for storing and restoring the result of a reconstruction.

 CALLING SEQUENCE:

       complex_array = NADIA_READ_CPLX( nx, ny, filename)

 INPUTS:

       nx:
             The number of pixels horizontally (need to be checked?)

       ny:
             The number of pixels vertically (need to be checked?)

       filename:
             A string containing the name of the file to read.

 OUTPUTS:
       complex_array:
             A 2D array of COMPLEX numbers.

 EXAMPLE:
       white_field = NADIA_READ_CPLX(1024, 1024, 'white_field_file.cplx')

\end{verbatim}






  
 
\subsection{NADIA\_READ\_DBIN}
\begin{verbatim}

 NAME:
       NADIA_READ_DBIN

 PURPOSE:
       Read a double binary file (a binary file of doubles).

 CALLING SEQUENCE:

       image = NADIA_READ_DBIN( nx, ny, filename)

 INPUTS:

       nx:
             The number of pixels horizontally (need to be checked?)

       ny:
             The number of pixels vertically (need to be checked?)

       filename:
             A string containing the name of the file to read.

 OUTPUTS:
       image:
             The image in the format of a 2D array of doubles.

 EXAMPLE:
       data = NADIA_READ_DBIN(1024, 1024, 'data_file.dbin')

\end{verbatim}






  
 
\subsection{NADIA\_READ\_PPM}
\begin{verbatim}

 NAME:
       NADIA_READ_PPM

 PURPOSE:
       Read a ppm file.

 CALLING SEQUENCE:

       image = NADIA_READ_PPM( nx, ny, filename)

 INPUTS:

       nx:
             The number of pixels horizontally (need to be checked?)

       ny:
             The number of pixels vertically (need to be checked?)

       filename:
             A string containing the name of the file to read.

 OUTPUTS:
       image:
             The image in the format of a 2D array of doubles.

 EXAMPLE:
       data = NADIA_READ_PPM(1024, 1024, 'data_file.ppm')

\end{verbatim}






  
 
\subsection{NADIA\_READ\_TIFF}
\begin{verbatim}

 NAME:
       NADIA_READ_TIFF

 PURPOSE:
       Read a tiff image file. Note that there are other IDL commands
       which perform this same function.

 CALLING SEQUENCE:

       image = NADIA_READ_TIFF( nx, ny, filename)

 INPUTS:

       nx:
             The number of pixels horizontally (need to be checked?)

       ny:
             The number of pixels vertically (need to be checked?)

       filename:
             A string containing the name of the file to read.

 OUTPUTS:
       image:
             The image in the format of a 2D array of doubles.

 EXAMPLE:
       data = NADIA_READ_TIFF(1024, 1024, 'data_file.tif')

\end{verbatim}






  
 
\subsection{NADIA\_SCALE\_INTENSITY}
\begin{verbatim}

 NAME:
       NADIA_SCALE_INTENSITY

 PURPOSE:
       Scale the intensity (magnitude squared) of the given complex
       array to the data. The intensity data must have been
       previously set using either one of the NADIA_INIT functions or
       NADIA_SET_INTENSITY.

       If Fresnel reconstruction is being done, the white-field will
       automatically be added to the complex array prior to scaling,
       and will be subtracted afterward.
       
       This function is called when using (NADIA_ITERATE) so generally
       won't need to be call explicitly. An exception to this is if
       the user wishes to extend their reconstruction with an
       addition constraint.


 CALLING SEQUENCE:

	result = NADIA_SCALE_INTENSITY( complex_array [,/SUPPRESS_DISPLAY] )

 INPUTS:

       complex_array:
             A 2D array of COMPLEX values which represents a wave in
             the detector plane.

 KEYWORD PARAMETERS:

       /SUPPRESS_DISPLAY:
             Do not display the result on the screen. This maybe useful
             if this function is used within a for loop. 

 OUTPUTS:

       result:
             A COMPLEX 2D array after the intensity has been scaled
             to data.

 EXAMPLE:
       See the example for NADIA_PROPAGATE_FROM_DETECTOR

\end{verbatim}






  
 
\subsection{NADIA\_SET\_ALGORITHM}
\begin{verbatim}

 NAME:
       NADIA_SET_ALGORITHM

 PURPOSE:
       Select the reconstruction algorithm to use. Options are:
       'ER' - error reduction 
       'BIO' - basic input-output 
       'BOO' - basic output-output 
       'HIO' - hybrid input-output 
       'DM' - difference map 
       'SF' - solvent-flipping 
       'ASR' - averaged successive reflections 
       'HPR' - hybrid projection reflection 
       'RAAR' - relaxed averaged alternating reflectors

 CALLING SEQUENCE:

       NADIA_SET_ALGORITHM, algorithm

 INPUTS:

       algorithm:
             Set the reconstruction algorithm. It should be
             one of the options listed above. Note that it is
             passed as a string.

 EXAMPLE:
       Perform 1000 iterations in total, changing to 
       error-reduction at the 400th iteration:
       ....
       a = NADIA_ITERATE(400)
       NADIA_SET_ALGORITHM, 'ER'
       a = NADIA_ITERATE(600)

\end{verbatim}






  
 
\subsection{NADIA\_SET\_BEAM\_STOP}
\begin{verbatim}

 NAME:
       NADIA_SET_BEAM_STOP

 PURPOSE: 
       For PlanarCDI, you may set the beam stop region in the
       detector plane. This region will be left to float (i.e. left
       unscaled) when the intensity scaling in performed. Zeros in
       the array (which is passed) indicate the beam-stop region,
       all other values indicate that those pixel should be scaled.

 CALLING SEQUENCE:

	NADIA_SET_BEAM_STOP, beam_stop_region

 INPUTS:

       beam_stop_region: 
             A 2D array of doubles or integers. Values of 0 are
             considered inside the beam-stop region. All others are
             considered to be outside the beam-stop region.

 EXAMPLE:

       nadia_set_beam_stop, my_beam_stop_regio


\end{verbatim}






  
 
\subsection{NADIA\_SET\_CHARGE\_FLIPPING}
\begin{verbatim}

 NAME:
       NADIA_SET_CHARGE_FLIPPING

 PURPOSE: For use with FresnelCDI reconstruction. Enabeling this
       procedure will constrain the transmission function phase to
       lie between -PI and 0. During each iteration (directly after
       applying the support constraint), the phase of the
       transmission function will be flipped if it is positive
       (i.e. if the phase, phi, lies between 0 and PI it will be
       reset to -phi). If this procedure is called for a PlanarCDI
       reconstruction, the contraint will be applied to the
       exit-surface-wave.

 CALLING SEQUENCE:

       NADIA_SET_CHARGE_FLIPPING, enable

 INPUTS:

       enable:
             This should be either 0 - turn off or 1 - turn on.

 EXAMPLE:
       NADIA_SET_CHARGE_FLIPPING, 1

\end{verbatim}






  
 
\subsection{NADIA\_SET\_CUSTOM\_ALGORITHM}
\begin{verbatim}

 NAME:
       NADIA_SET_CUSTOM_ALGORITHM

 PURPOSE:
       Set a custom reconstruction algorithm.
       
       Iterative reconstruction algorithms can be expressed as a
       combination of several operators. The parameters to this
       procedure set the coefficients for combinations of these
       operators. For a description, please see page 22 of the
       H.M. Quiney review: TUTORIAL REVIEW, Coherent diffractive
       imaging using short wavelength light sources, Journal of
       Modern Optics, 2010, DOI: 10.1080/09500340.2010.495459.  Some
       description is also given in the C++ doxygen documentation
       for PlanarCDI::set_custom_algorithm.


 CALLING SEQUENCE:

       NADIA_SET_CUSTOM_ALGORITHM, m1, m2, m3, m4, m5, m6, m7, m8, m9, m10

 INPUTS:

       m1-m10:
             Coefficients to the operator combinations.

 EXAMPLE:
       Perform 1000 iterations in total, changing to a custom
       algorithm at the 400th iteration and print the algorithm to screen:

       ....
       a = NADIA_ITERATE(400)
       NADIA_SET_ALGORITHM, 0.5, 0.5, 0.5, 0.5, 0.5, $ 
                          0.5, 0.5, 0.5, 0.5, 0.5
       NADIA_PRINT_ALGORITHM                    
       a = NADIA_ITERATE(600)

\end{verbatim}






  
 
\subsection{NADIA\_SET\_INTENSITY}
\begin{verbatim}

 NAME:
       NADIA_SET_INTENSITY

 PURPOSE:
       Set the detector intensity data. This will override 
       the intensity given to any of the NADIA_INIT methods.
       In general, users should not need to call this method.

 CALLING SEQUENCE:

	NADIA_SET_INTENSITY, data

 INPUTS:

       data: 
             The detector data. It should be in the form of a 2D
             array or doubles or integers.


 EXAMPLE:

       nadia_set_intensity, my_data


\end{verbatim}






  
 
\subsection{NADIA\_SET\_RELAXATION\_PARAMETER}
\begin{verbatim}

 NAME:
       NADIA_SET_RELAXATION_PARAMETER

 PURPOSE:
       Set the relaxation parameter. The default relaxation
       parameter used in Planar and Fresnel reconstruction is 0.9.
       In Fresnel white-field reconstruction, this parameter is 
       not used.

 CALLING SEQUENCE:

	NADIA_SET_RELAXATION_PARAMETER, beta

 INPUTS:

       beta: 
             The relaxation parameter.

 EXAMPLE:

       nadia_set_relaxation_parameter, 0.9

\end{verbatim}






  
 
\subsection{NADIA\_SET\_SUPPORT}
\begin{verbatim}

 NAME:
       NADIA_SET_SUPPORT

 PURPOSE:
       Set the support shape to be used in reconstruction. This with override 
       the support given to any of the NADIA_INIT methods and maybe called
       at any time during the reconstruction.  

 CALLING SEQUENCE:

	NADIA_SET_SUPPORT, support

 INPUTS:

       support: 
             A 2D array of doubles or integers giving the sample's 
             (or zone-plate's) support. Values of 1 or greater are 
             considered inside the support. All others are considered 
             to be outside the support.

 EXAMPLE:

       nadia_set_support, my_support


\end{verbatim}






  
 
\subsection{NADIA\_SET\_TRANS\_UNITY\_CONSTRAINT}
\begin{verbatim}

 NAME:
       NADIA_SET_TRANS_UNITY_CONSTRAINT

 PURPOSE: For use with FresnelCDI reconstruction. Enabeling this
       procedure will constrain the transmission function magnitude
       to lie 0 and 1. During each iteration (directly after applying
       the support constraint), the magnitude of the transmission
       function will be reset to 1 at locations where it is greater
       than 1.

 CALLING SEQUENCE:

       NADIA_SET_TRANS_UNITY_CONSTRAINT, enable

 INPUTS:

       enable:
             This should be either 0 - turn off or 1 - turn on.

 EXAMPLE:
       NADIA_SET_TRANS_UNITY_CONSTRAINT, 1

\end{verbatim}

  
 
\subsection{NADIA\_WRITE\_CPLX}
\begin{verbatim}

 NAME:
       NADIA_WRITE_CPLX

 PURPOSE:
       Write a binary file of fftw complex numbers. This is
       useful for storing and restoring the result of a reconstruction.

 CALLING SEQUENCE:

       NADIA_WRITE_CPLX( complex_array, filename )

 INPUTS:

       complex_array:
             A 2D array of COMPLEX numbers

       filename:
             A string containing the name of the file to write.

 EXAMPLE:
       NADIA_WRITE_CPLX, white_field, 'white_field_file.cplx'

\end{verbatim}

  
 
\subsection{NADIA\_WRITE\_DBIN}
\begin{verbatim}

 NAME:
       NADIA_WRITE_DBIN

 PURPOSE:
       Write a binary file of doubles. 

 CALLING SEQUENCE:

       NADIA_WRITE_DBIN( image, filename )

 INPUTS:

       image:
             A 2D array of numbers

       filename:
             A string containing the name of the file to write.

 EXAMPLE:
       NADIA_WRITE_DBIN, result, 'reco_magnitude.dbin'

\end{verbatim}

